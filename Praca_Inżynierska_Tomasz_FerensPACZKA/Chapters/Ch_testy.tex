% Chapter Ocena działania i skuteczności urządzenia
\chapter{Testowanie} % Chapter title

Jednym z kluczowych zagadnień każdego projektu jest weryfikacja poprawności działania urządzenia. Oprócz weryfikacji funkcjonalnej w pracy tej istotne są pomiary czasów odpowiedzi poszczególnych modułów, a przede wszystkim modułu Object\_Detection odpowiedzialnego za obsługę kamery i przetwarzanie obrazu. Pomiary te pozwolą oszacować całkowity czas odpowiedzi urządzenia, a także dzięki nim będzie można porównać działanie układu opartego na platformie Raspberry i kamerze dedykowanej oraz komputerze klasy PC z kamerą USB.
Na potrzeby procesu testowania zaimplementowane zostały odpowiednie moduły testujące, umożliwiające dzięki zastosowanej przy projektowaniu aplikacji technice OO(orientowanej-obiektowo) w wygodny sposób przeprowadzić scenariusz testowy zaprezentowany na rys.<rysunek>.
\section{Implementacja modułu testowego}
Do przetestowania działania funkcji oraz pomiaru czasu  stworzony został oddzielny moduł Time\_Test. Moduł ten wykorzystuje implementację obiektowo-orientowaną aplikacji głównej, co pozwala w łatwy i przejrzysty sposób dokonać testu i pomiaru czasu działania każdego modułu roboczego.

\begin{lstlisting}[caption = {Klasa testująca}, label=TestClass, language=C++]
class Time_Test{
private:
	int module_nr;
	std::vector <Module*> wektor;

public:
	void add(Module **);
	void measure_time();
	void display_results();
	}; 
\end{lstlisting}

\begin{description}
\item Praca modułu opiera się na trzech funkcjach:
\begin{itemize}[noitemsep]
\item void add(Module**) – metoda ta przyjmuje jako argument tablicę wskaźników na obiekty typu Module i umieszcza je w składowej wektor.
\item bool measure\_time() – metoda odpowiedzialna za pomiar czasów poszczególnych modułów oraz całkowitego. 
\end{itemize}
\item Metoda ta wykorzystuje dostarczone wraz z biblioteką standardową języka C++ bibliotekę chrono, która opiera się na 3 głównych obiektach, jak : 
\begin{itemize}[noitemsep]
\item duration – mierzące okres czasu – w projekcie są to globalne zmienne : time\_span (czas działania modułu), total\_time (czas całkowity) oraz wektor czasów times przechowujący poszczególne czasy działania modułów.
\item time\_point – odniesienie do punktu w czasie – w projekcie są to zmienne t1 oraz t2
\item clock – framework, odpowiadający za odniesienie czasu określonego przez obiekty typu time\_point względem czasu rzeczywistego. – w projekcie są to obiekty typu  time\_point oraz metoda now() zwracająca aktualny czas frameworku high\_resolution\_clock, korzystającego z zegara wysokiej precyzji.

\end{itemize}
\end{description}


\begin{lstlisting}[caption = {Listing Implementacja funkcji testującej}, label=TestFun, language=C++]
void Time_Test::measure_time(int n){

	printf("Start of time tests \n");

	high_resolution_clock::time_point t1;
	high_resolution_clock::time_point t2 ;

	duration<double>  time_span,total_time;
	std::vector <std::chrono::duration<double>> times;
           n=N;
	printf("\nNumber of iterations : %d",n);

		for (int i=0; i<module_nr; ++i){
			 t1 = high_resolution_clock::now();
c=0;
while(c<N){
		 assert(! wektor[i]->work());
c++;
}	
		 	t2 = high_resolution_clock::now();

			 time_span  = duration_cast<nanoseconds>(t2 - t1);
			 times.push_back(time_span);
			 total_time += time_span;
		  } 

\end{lstlisting}

Funkcja przedstawiona na listingu 5.2 posiada pętlę główną for – w ramach której następuje iteracja po kolejnych modułach i wywołanie polimorficzne funkcji work() dla każdego z nich. Wywołanie to jest objęte dodatkową pętlą while, która jest odpowiedzialna za uśrednienie wyników pomiaru czasu. W każdej iteracji pętli for funkcja ta zapisuje w zmiennych t1 i t2 punkty czasowe odpowiednio przed i po wywołaniu N-razy (n jest argumentem wejściowym wywołania programu głównego) funkcji składowej work() badanego modułu, a następnie zapisuje wartość ich różnicy do zmiennej time\_span. Czas zmierzony dla każdego modułu jest umieszczony w wektorze times. Jednocześnie w zmiennej total\_time przechowywany jest czas całkowity wykonania sekwencji modułów.

Metoda ta dokonuje także testu poprawności działania funkcji. Polega to na wywołaniu każdej funkcji roboczej work() modułów wewnątrz funkcji assert().
Jeśli funkcja work zwróci wartość różną od 0 to funkcja assert poda komunikat wraz z nazwą pliku źródłowego i numeru linii w której to się stało.

Za pomocą metody display\_results() wyniki  przechowywane w wektorze times i zmiennej total\_time są uśrednione i prezentowane użytkownikowi dla każdego modułu.

\section{ Wyniki pomiarów}
W tabeli 5.1 znajduje się zestawienie pomiarów czasów wykonanych na platformie Raspberry Pi 2 z modułem RaspiCam oraz na komputerze klasy notebook: ASUS  X53S z modułem kamery USB.
Specyfikacja platformy ASUS:
\begin{itemize}[noitemsep]
\item Procesor Inter Core i5 2410M
\item Pamięć 8GB 1333MHz DDR3
\item GPU: Nvidia GeForce GT 540M 1GB DDR3
\item Dysk twardy 750GB SATA 7200 obr./min.
\end{itemize}
Systemem wykorzystanym na platformie ASUS jest Linux Ubuntu 14.04 uruchamianym w maszynie wirtualnej programu VirtualBox (wersja 4.3.30). 
Zasoby przydzielone maszynie wirtualnej :
\begin{itemize}[noitemsep]
\item 4GB pamięci RAM
\item 2 wirtualne rdzenie CPU
\item 128 MB Video Memory
\end{itemize}
Specyfikacja platformy Raspberry :
\begin{itemize}[noitemsep]
\item Procesor BCM 2836  quad core
\item 1GB pamięci RAM DDR2
\item GPU: VideoCore IV (przydzielone 128 MB pamięci RAM z zasobów wspólnych)
\end{itemize}
Poniżej znajdują się wyniki dla N = 10, 100 i 500 powtórzeń wywołania funkcji roboczych poszczególnych modułów dla rozdzielczości strumienia : 320x240.

\begin{table}[hbt!]
%\myfloatalign
\caption[Ilość iteracji N = 10]{Zestawienie wyników pomiarów czasu dla różnych wartości N. Ilość iteracji N = 10}
\begin{tabularx}{\textwidth}{|l|X|X|} 
 \hline
Moduł&	ASUS x53S&	Raspberry Pi 2\\ \hline
Object\_Detection [ms]	&254,762741&	549,6572\\
Worker [us]	&832,679&	98,599\\
Logger [ms]	&0,712403&	6,425799\\
Całkowity średni czas [ms]&	256&	556 \\
\hline
\end{tabularx}  
\label{tab:compareAnalysers}
\end{table}


\begin{table}[hbt!]
%\myfloatalign
\caption[Ilość iteracji N = 100]{Zestawienie wyników pomiarów czasu dla różnych wartości N. Ilość iteracji N = 100}
\begin{tabularx}{\textwidth}{|l|X|X|} 
 \hline
Moduł&	ASUS x53S	&Raspberry Pi 2 \\ \hline
Object\_Detection [ms]&	137,7835103	&523,30284 \\
Worker [us]	&387,455	&80,37 \\
Logger [ms]&	1,907358&	6,440570 \\
Całkowity średni czas [ms]	&140&	529 \\
\hline
\end{tabularx}  
\label{tab:compareAnalysers}
\end{table}


\begin{table}[hbt!]
%\myfloatalign
\caption[Ilość iteracji N = 500]{Zestawienie wyników pomiarów czasu dla różnych wartości N. Ilość iteracji N = 500}
\begin{tabularx}{\textwidth}{|l|X|X|} 
 \hline
Moduł&	ASUS x53S	&Raspberry Pi 2\\ \hline
Object\_Detection [ms]&	118,555288	&525,432592\\
Worker [us]&	95,112&	77,03\\
Logger [ms]&	1,339017&	6,264304\\
Całkowity średni czas [ms]&	119&	531\\
\hline
\end{tabularx}  
\label{tab:compareAnalysers}
\end{table}

\section{Uruchomienie testów}
\begin{description}
\item Na proces testowania składa się:
\begin{itemize}[noitemsep]
\item Zbudowanie projektu 
\item Załadowanie modułu sterownika V4L2
\item Uruchomienie aplikacji testowej z odpowiednimi argumentami
\end{itemize}
\end{description}
Na zbudowanie projektu składają się min. Kompilacja warunkowa ( zbędne do testowania bloki aplikacji głównej są pomijane), oraz wywołanie skryptu budującego makefile.
Załadowanie sterownika możliwe jest z poziomu linii komend i musi być koniecznie wykonane przed uruchomieniem aplikacji. Minimalna wersja komendy ta jest następująca:\\
uv4l $--$driver raspicam $--$video\_nr nr\\
Gdzie nr to index pod jakim zostanie zarejestrowane urządzenie RaspiCam przez sterownik V4L2 w systemie (np. dla nr = 0 będzie to : /dev/video0).\\
Dodatkowe argumenty pozwalające konfigurować sensor z poziomu sterownika to :\\
$--$encoding yuv420\\
$--$width 320\\
$--$height 240\\
Atrybut $--$encoding pozwala wymusić format kodowania obrazu, w tym przypadku zastosowany jest format YUV420. Kolejne dwa argumenty są odpowiedzialne za szerokość i wysokość obrazu strumienia(w projekcie można je pominąć z  uwagi na to, że z poziomu aplikacji także są konfigurowalne, co jest opisane niżej oraz w rozdziale 4.).\\  
Uruchomienie wymaga podania ścieżki pliku binarnego cam wraz z argumentami oddzielonymi spacjami:\\
./cam width height N\\
width – szerokość klatki strumienia video\\
height – wysokość klatki strumienia video\\
N – ilość iteracji ( opisane w paragrafie 5.2)\\
Przykładowe wywołanie : ./cam 320 240 100\\
Uruchomi w rezultacie aplikację testową z parametrami strumienia : rozdzielczość 320x240 oraz wykona uśrednione na 100 iteracjach testy.

\section{Wnioski}
Na podstawie pomiarów stwierdzić można, że kamera zrealizowana na platformie Raspberry ustępuje urządzeniu o dużo większej mocy obliczeniowej jakim jest laptop ASUS x53S.
Jak widać przy większej ilości iteracji średnie czasy są znacznie mniejsze. W przypadku platformy ASUS wyraźnie widać względny spadek czasów. Najprawdopodobniej wynika to z optymalizacji sprzętowej procesora i jednostki GPU przy wielokrotnym odnoszeniu się do tego samego fragmentu pamięci.
W obu przypadkach na całkowity czas odpowiedzi kamery ma wpływ głównie działanie modułu Object\_Detection. Nie jest to zaskakujący rezultat, gdyż moduł ten wykonuje dwa ważne zadania tj. pośredniczy w przechwyceniu strumienia video oraz dokonuje przetworzenia każdej przez algorytm wykorzystujący kaskady Haar’a. Zdecydowana przewaga platformy ASUS wynika z dużo większej wydajności jednostki GPU i CPU względem platformy Raspberry. Dla pomiaru czasu działania modułu Logger’a kluczowym aspektem jest nośnik na jakim dokonywany jest zapis zdjęcia wykrytego obiektu. 
Raspberry Pi 2 wykorzystuje kartę Micro SD class 10, co wg specyfikacji powinno gwarantować minimalny transfer na poziomie 10MB/s. W przypadku ASUS’a jest to dużo szybszy dysk twardy wykorzystujący magistralę SATA, zapewniającą wg specyfikacji transfer na poziomie ok. 179MB/s. Stąd też widoczna jest róznica w czasach.
Jako, że laptop ASUS nie posiada portów GPIO, test modułu Worker w zasadzie pokazuje tylko czas wywołania funkcji roboczej.\\
Podsumowując – platforma Raspberry pod względem wydajności ustępuje rozwiązaniom desktopowym, jednak różnica nie jest ogromna. Czas reakcji na poziomie 0,5s z powodzeniem pozwala na wykorzystanie urządzenia w wielu dziedzinach.
\begin{description}
\item Ponadto uwzględniając atuty takie jak:
\begin{itemize}
 \item niewielkie gabaryty
 \item niskie zużycie energii
 \item możliwość sterowania urządzeniami zewnętrznymi 
 \item mobilność - łatwość montażu w warunkach trudno dostępnych
 \item niską cenę
 \item możliwość własnej implementacji sprzętowej (przygotowanie dedykowanego urządzenia na podstawie płyty bazowej)
  
 Raspberry Pi 2 zdecydowanie wygrywa z platformą ASUS.
\end{itemize}
\end{description}

%Podsumowanie ma być jako nowy rozdział (chapter!)
\chapter{Podsumowanie} 

Realizacja pracy zakończyła się sukcesem. Kamera spełnia wszystkie założenia i jest zgodna z obraną koncepcją. Potwierdzeniem tego są wykonane testy, które zweryfikowały projekt jako poprawnie działający, a do tego pokazały realne możliwości sprzętowo-programowe urządzenia.\\
Niniejsza praca jest doskonałą bazą wyjściową do dalszego rozwoju. Dzięki zaimplementowanej elastycznej strukturze aplikacji możliwe jest rozszerzenie funkcjonalności o komunikację z innymi urządzeniami i stworzenie sieci kamer współpracujących ze sobą. A także adaptację innych algorytmów na potrzeby np. diagnostyki medycznej, czy zastosowań w robotyce.