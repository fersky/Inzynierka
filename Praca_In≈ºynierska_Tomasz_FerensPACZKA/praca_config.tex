%dużo bibliotek ale niech zostają nie chce mi się poprawiać
\usepackage{graphicx}
\usepackage{caption}
\usepackage{subcaption}
\usepackage{amsfonts}
\usepackage{amsmath}
\usepackage{amsthm}
\usepackage{wrapfig}
%\usepackage{listings}
\usepackage{xcolor}
%\lstset{language=C}
\usepackage{fancyhdr}

\usepackage{moreverb}
\usepackage{booktabs}
\usepackage{eurosym}
\usepackage{indentfirst}
\usepackage{pdfpages}
\usepackage[utf8]{inputenc}
\usepackage{enumitem}
\usepackage{multirow}
\usepackage{rotating}
\usepackage{float}
\usepackage{tocloft}
\usepackage{color}
%\usepackage{xcolor,colortbl}
%\usepackage{mathtools}
%\usepackage{tablefootnote}
\usepackage{setspace}\setstretch{1.15} %było 1.2
\usepackage{tabularx}
%\usepackage{tgtermes}
\usepackage[titletoc]{appendix}
\usepackage{booktabs}
\usepackage{enumitem}
\setlist[1]{itemsep=-5pt}
\newcommand{\ra}[1]{\renewcommand{\arraystretch}{#1}}
\usepackage{color, colortbl}
\usepackage{array}
\usepackage{ragged2e}
\newcolumntype{P}[1]{>{\RaggedRight\hspace{0pt}}p{#1}}
%----------------------------------------------------------------------------------------
%	HYPERREFERENCES
%----------------------------------------------------------------------------------------
\usepackage{hyperref}  % backref linktocpage pagebackref

\usepackage{tabularx} % Better tables

\usepackage{enumitem} % Required for manipulating the whitespace between and within lists

\usepackage{polski}
\usepackage[utf8]{inputenc}

% LISTING
\usepackage{listings}
\usepackage{color}

\definecolor{mygreen}{rgb}{0,0.6,0}
\definecolor{mygray}{rgb}{0.5,0.5,0.5}
\definecolor{mymauve}{rgb}{0.58,0,0.82}

\lstset{ %
  backgroundcolor=\color{white},   % choose the background color; you must add \usepackage{color} or \usepackage{xcolor}
  basicstyle=\footnotesize,        % the size of the fonts that are used for the code
  breakatwhitespace=false,         % sets if automatic breaks should only happen at whitespace
  breaklines=true,                 % sets automatic line breaking
  captionpos=b,                    % sets the caption-position to bottom
  commentstyle=\color{mygreen},    % comment style
  deletekeywords={...},            % if you want to delete keywords from the given language
  escapeinside={\%*}{*)},          % if you want to add LaTeX within your code
  extendedchars=true,              % lets you use non-ASCII characters; for 8-bits encodings only, does not work with UTF-8
  frame=single,                    % adds a frame around the code
  keepspaces=true,                 % keeps spaces in text, useful for keeping indentation of code (possibly needs columns=flexible)
  keywordstyle=\color{blue},       % keyword style
  language=VHDL,                 % the language of the code
  morekeywords={*,...},            % if you want to add more keywords to the set
  numbers=left,                    % where to put the line-numbers; possible values are (none, left, right)
  numbersep=5pt,                   % how far the line-numbers are from the code
  numberstyle=\tiny\color{mygray}, % the style that is used for the line-numbers
  rulecolor=\color{black},         % if not set, the frame-color may be changed on line-breaks within not-black text (e.g. comments (green here))
  showspaces=false,                % show spaces everywhere adding particular underscores; it overrides 'showstringspaces'
  showstringspaces=false,          % underline spaces within strings only
  showtabs=false,                  % show tabs within strings adding particular underscores
  stepnumber=2,                    % the step between two line-numbers. If it's 1, each line will be numbered
  stringstyle=\color{mymauve},     % string literal style
  tabsize=2,                       % sets default tabsize to 2 spaces
  title=\lstname                   % show the filename of files included with \lstinputlisting; also try caption instead of title
}


\renewcommand\lstlistingname{Wydruk}
\renewcommand\lstlistlistingname{Wydruki}
\renewcommand\listfigurename{{\LARGE Rysunki}}
\renewcommand{\figurename}{Rys.}
\renewcommand\tablename{Tab.}

\renewcommand{\listtablename}{{\LARGE Tablice}}
\renewcommand{\bibname}{Literatura}


%    Abstract: \abstractname
%    Appendix: \appendixname
%    Bibliography: \bibname
%    Chapter: \chaptername
%    Contents: \contentsname
%    Index: \indexname
%    List of Figures: \listfigurename
%    List of Tables: \listtablename
%    Part: \partname
%    References: \refname

% Nie pozostawia na końcu wdów i sierot
\usepackage{lettrine}\sloppy %zakaz wydłużania lini (gdzy nie może złożyć) 
\clubpenalty=10000 %nie dieli wyrazów pomiędzy stronami 
\widowpenalty=10000 %nie pozostawia wdów i sierot pojedynczych 
\usepackage{fancyhdr}

\usepackage{pdfpages}

\usepackage{graphicx}
\usepackage{caption}
\usepackage{subcaption}

%---------------------------------------------------------------