% Abstract
\newpage
\pdfbookmark[1]{Abstract}{Abstract} % Bookmark name visible in a PDF viewer

\begingroup
\let\clearpage\relax
\let\cleardoublepage\relax
\let\cleardoublepage\relax

\hfill \\
\vspace{0.1cm} 
\hfill \\
%\chapter*{Streszczenie} % Abstract name
{\LARGE \textbf{Streszczenie \\ \vspace{0.5cm} \\ }}

Tytuł: \textbf{\textit{Szybka, Inteligentna kamera z interfejsem Ethernet.}} \\

\textit{W poniższej pracy zaprezentowany jest projekt szybkiej, inteligentnej kamery opartej na systemie wbudowanym Raspberry Pi 2 z układem SoC BCM2836 firmy Broadcom oraz modułem kamery RaspiCam -  sensor ov5647. 
Dzięki wydajnym podzespołom płyty Raspberry Pi 2 możliwe jest szybkie przetworzenie obrazu w celu detekcji wymaganego wzorca i wykonanie określonej akcji.
Moduł Ethernet umożliwia komunikację między użytkownikiem, a płytą za pomocą sieci internetowej.
Płyta pracuje pod kontrolą systemu Linux, dystrybucji Raspbian , na którym uruchamiana jest aplikacja napisana w języku C++.
W pracy przedstawiona jest jedna z wielu koncepcji wykorzystania inteligentnej kamery oraz jej praktyczna realizacja.} \\

Słowa kluczowe: \textbf{\textit{systemy wbudowane, wizja komputerowa, Ethernet, Internet of Things, Raspberry Pi 2, OV5647, OpenCV, Linux, przetwarzanie sygnałów. }} 
\\
\vspace{1cm} \\

{\LARGE \textbf{Abstract \\ \vspace{0.5cm} \\ }}

Title: \textbf{\textit{Fast, smart camera with Ethernet interface.}} \\

\textit{In this paper is presented project fast, smart camera based on embedded system Raspberry Pi 2 with SoC Broadcom BCM2836 and camera module RaspiCam - ov5647 sensor.
Thanks to the efficient hardware components of Raspberry Pi 2, it is possible to quickly process the image in order to detect the desired pattern and perform specific action.
Ethernet module enables communication between the user and the device via the Web.
The device is running under control of Linux system, distribution Raspbian used to run an application written in C ++.
The idea presented in this work is one of the many concepts of using smart camera and its practical implementation.} \\

Key words: \textbf{\textit{Embedded Systems, Digital Image Processing, Ethernet, Internet of Things, Raspberry Pi 2, OV5647, OpenCV, Linux, Signal Processing.}}

\endgroup			

\newpage